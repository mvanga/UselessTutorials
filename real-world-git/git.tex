\chapter{Git}

Git is a distributed version control system.

\section{Creating a New Repository}

To create a new repository, you can use the \emph{init} command of git. You 
can use it as follows:

\begin{lstlisting}[label=git-init, caption='Initializing a git repository']
$ cd $FOLDER_WITH_SOURCE_CODE
$ git init
\end{lstlisting}

Git stores everything into a hidden folder in the top-level directory
named \emph{.git}.

\section{Destroying an Existing Repository}

There may be times when you want to get rid of a git repository. The 
easiest way to do this is to just remove the \emph{.git} folder.

\begin{lstlisting}[label=git-destroy, caption='Destroying a git repository']
$ cd $FOLDER_WITH_GIT_REPO
$ rm -rf .git
\end{lstlisting}

\section{Adding Files to a Repository}

Once you have created a repository, you can add specific files to it 
using the \emph{add} command as shown below.

\begin{lstlisting}[label=git-add, caption='Adding files to a git repository']
$ touch file.txt
$ git add file.txt
\end{lstlisting}
