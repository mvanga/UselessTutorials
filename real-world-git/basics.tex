\chapter{The Basics}

Git is a distributed version control system.

\section{Creating a New Repository}

To create a new repository, you can use the \emph{init} command of git. You 
can use it as follows:

\begin{lstlisting}[label=git-init, caption='Initializing a git repository']
$ cd $FOLDER_WITH_SOURCE_CODE
$ git init
\end{lstlisting}

Git stores everything into a hidden folder in the top-level directory
named \emph{.git}.

\section{Destroying an Existing Repository}

There may be times when you want to get rid of a git repository. The 
easiest way to do this is to just remove the \emph{.git} folder.

\begin{lstlisting}[label=git-destroy, caption='Destroying a git repository']
$ cd $FOLDER_WITH_GIT_REPO
$ rm -rf .git
\end{lstlisting}

\section{Adding Files to the Index}

You can add specific files to the index (see below) using the \emph{add} 
command as shown below.

\begin{lstlisting}[label=git-add, caption='Adding files to a git repository']
$ vim test.c
$ git add test.c
\end{lstlisting}

There are three main concepts that can help you visualize the process
used by Git. These are part of every single Git repository:

\begin{description}
\item[Repository:] This consists of all commited changes. When you
commit something, it is added to the current repository.
\item[Working Directory:] The working directory consists of all the
uncommited changes made since the last commit in the current repository.
\item[Index:] This is all the things that will be commited into the next
commit.
\end{description}

\begin{itemize}
\item When you do the \emph{init} command, you are in effect, creating an
empty repository.
\item When you edit a file and save it, the changes you have made get 
added to the working directory.
\item When you use the \emph{add} command on the file you just edited, 
the changes you made are added to the index.
\item When you commit the changes using the \emph{commit} command, only
the changes in the index are moved to the repository. The changes are
also removed from the working directory.
\end{itemize}

Having these three distinct parts of the process has many advantages. For
example, support you have made many changes and want to organize these changes
into multiple commits for clarity. All your uncommited changes are automatically
in your working directory. You can now add a specific file or part of a 
file (see section \ref{sec:add-i}) and only those changes will be added to the
index. Now you can commit the changes and redo the process for the rest
of the changes.

\section{Adding Parts of a File to the Index}
\label{sec:add-i}
There are times when you do not want to commit all the changes you
made to a file into one commit. In this case, you need a way to add
specific lines within a file to the Git index. You can do this using
the \emph{interactive} mode of the \emph{add} command:

\begin{lstlisting}[label=git-add-i, caption='Adding parts of files to a git repository']
$ git add -i
           staged     unstaged path
  1:    unchanged        +4/-0 [t]est.c

*** Commands ***
  1: [s]tatus     2: [u]pdate     3: [r]evert     4: [a]dd untracked
  5: [p]atch      6: [d]iff       7: [q]uit       8: [h]elp
What now>
\end{lstlisting}

You should get something like the above when you run this command. Now
select the \emph{patch} mode by typing '5' or 'p'. You should get a new
prompt like the one below:

\begin{lstlisting}[label=git-add-i, caption='Adding parts of files to a git repository']
What now> p
           staged     unstaged path
  1:    unchanged        +4/-0 [t]est.c
Patch update>> 
\end{lstlisting}

You can select the files, from which you want changes to be added, by 
their number.

\begin{lstlisting}[label=git-add-i, caption='Adding parts of files to a git repository']
Patch update>> 1
           staged     unstaged path
* 1:    unchanged        +4/-0 [t]est.c
\end{lstlisting}

An asterisk should appear next to the selected file. Once you have
done this for all the files you want to add parts from, just press 
return to begin adding:

\begin{lstlisting}[label=git-add-i, caption='Adding parts of files to a git repository']
Patch update>> 
diff --git a/test.c b/test.c
index 691cfab..d148f17 100644
--- a/test.c
+++ b/test.c
@@ -8,3 +8,7 @@ int main(int argc, char *argv[])
 	printf("Hello World %d\n", i);
	return 0;
}
+
+// one commit
+
+// 2nd commit
Stage this hunk [y,n,q,a,d,/,e,?]? 
\end{lstlisting}

At this point you can press \emph{'y'} to add this part of the file (called
a hunk for some weird reason) to the index. You can select \emph{'n'} to 
not add it to the index. If you want to break it into smaller parts,
you can choose \emph{'s'}.

Once you are done, you just enter \emph{'q'} to ignore the rest and return
to the shell. Only parts of the file would have been added.
\section{Changing commits from the past}
Let's assume that you just have made several commits and after that you noticed that accidently one of them added some debugging printing function calls. They were very helpful at the stage of development but now they are useless.
The trivial way of solving this problem would be to simply remove the calls and commit the fixed code. Luckily, you may avoid  cluttering the repository with this kind of commits.
Let's say that you want to edit code from the last but three commit. Type then: 
\begin{lstlisting}
git rebase -i HEAD~3
\end{lstlisting}

HEAD~3 stays for editing three commits from the HEAD, which is the last commit. -i stays from interactive mode. After this command you will see a text like the following.

\begin{lstlisting}
pick b5e5583 Check if XILINX env variable is set
pick 11edaf6 Handle unexpected parsing errors in manfiests
pick b6efcc6 Remove debugging prints

# Rebase 7d63e57..b6efcc6 onto 7d63e57
#
# Commands:
#  p, pick = use commit
#  r, reword = use commit, but edit the commit message
#  e, edit = use commit, but stop for amending
#  s, squash = use commit, but meld into previous commit
#  f, fixup = like "squash", but discard this commit's log message
#
# If you remove a line here THAT COMMIT WILL BE LOST.
# However, if you remove everything, the rebase will be aborted.
#
\end{lstlisting}
You see there a list of available commands. Delete ``pick'' at the screwed commit, type ``edit'', save and exit. Now the code is brought back to the state from the selected commit. You can find the unneeded messages and remove them from your code. When you are done, follow the instructions that were shown in the console.

For amending the changes type
\begin{lstlisting}
git commit --amend
\end{lstlisting}
When you are satisfied with your changes, you commited them and want to go back to the last commit, type
\begin{lstlisting}
git rebase --continue
\end{lstlisting}
\vspace{1cm}
The rebase command may be also helpful when you want to fix a commit message from the past. Type then
\begin{lstlisting}
$ git rebase -i HEAD~3
\end{lstlisting}
You will see the editor with text that has to be edited. Get rid of ``pick'' next to you chosen commit and type ``r'' or ``reward''. After you save and exit, you can edit the commit message. Voilà!
\section{Viewing the Status of the Repository}
